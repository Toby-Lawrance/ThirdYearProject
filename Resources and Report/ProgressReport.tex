\documentclass[10pt,a4paper]{report}
\usepackage{fancyhdr}
\usepackage{lastpage}

% Title Page
\title{Third Year Project Progress Report}

\pagestyle{fancy}
\fancyhf{} % clear all fields

\fancyhead[R]{Toby Lawrance, 1731636, \thepage / \pageref{LastPage}}
\renewcommand{\headrulewidth}{0pt}

\begin{document}
\maketitle
\section*{Project Introduction/Specification}
	\subsection*{Statement}
		\subsubsection*{In Short:}
			Implementing Computer Vision algorithms to allow for robot navigation using the camera feed as the primary perception.
		\subsubsection*{Aims:}
			The aim of this project is to utilise the camera more efficiently as cameras are typically cheap sensors and this helps make robotics more affordable as a discipline, or at least provide additional ways to aid internal sensor data and provide additional data points of belief in the robot's current pose.
		\subsubsection*{Objectives:}
			\begin{itemize}
				\item Move towards an object that is recognised in the environment. 
				\item Acknowledge obstacles in the environment and attempt to avoid them.
				\item Attempt to acknowledge more natural obstacles.
				\item Attempt to use the camera feed to aid with pose estimation.
			\end{itemize}
\section*{Progress made so far}
	\subsection*{ROS}
		\subsubsection*{Overview of the Robot Operating System}
			The Robot Operating System (ROS) is a meta-operating system that runs on top of a regular operating system. It is designed to be highly modular (though one could at times argue too modular) and typically run in a "Publisher -> Subscriber" model, a unit of executable is referred to as a node and is typically recommended as a reusable task, for example "sensor drive, sensor data conversion, obstacle recognition, motor drive, encoder input and navigation"\cite{pyo_ros_en_2017}
\section*{Future plans}

\section*{Management}

\bibliographystyle{plain}
\bibliography{ProgressReport}

\end{document}